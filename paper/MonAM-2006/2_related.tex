\section{Related Work}
\label{sec:related-work}

%Network monitoring tools have a long history in the communications domain. 
%Usually, such approaches were integrated into special-purpose applications for accounting or traffic engineering environments. 
%With the development of the Netflow protocol by Cisco, a standardized method emerged that allows a strict separation of monitoring and analysis. The IETF recognized the need for an Internet-wide standard for collecting and transporting monitoring data. Therefore, NetFlow.v9, IPFIX and PSAMP are being standardized as RFCs.

In this section, we provide a brief overview to open-source implementations of monitoring probes.
There are several implementations supporting the NetFlow.v9 format, e.g. nprobe~\cite{deri2003nprobe} and NetMate~\cite{schmoll2004netmate}. Currently, the authors of these tools are working on IPFIX compliant versions.
Table~\ref{tab:features} compares the supported features of these implementations and Vermont.
Futhermore, there exist implementations from Cisco Systems and IBM~\cite{ibm-ipfix}, which are not available under an open-source compatible license and as such not listed here.

Vermont, being a reference implementation for both the IPFIX and the PSAMP standard, already supports IPFIX, PSAMP, and IPFIX aggregation schemes.

\begin{table}
%% increase table row spacing, adjust to taste
\renewcommand{\arraystretch}{1.0}
\caption{Feature Comparison}
\label{tab:features}
\begin{center}
%% Some packages, such as MDW tools, offer better commands for making tables
%% than the plain LaTeX2e tabular which is used here.
\begin{tabular}{r|c|c|c}
 & Vermont & nProbe & NetMate\\
\hline
IPFIX Support & yes & planned & \\
PSAMP Support & yes &  & \\
IPFIX Aggregation & yes & planned & planned\\
Collector Functionality & yes &  & \\
Save Data To Disk &  & yes & \\
Save Data To SQL DB & yes & & \\
Remote Reconfigurability & yes &  & yes\\
Plugin Architecture &  & yes & yes\\
\hline
\end{tabular}
\end{center}
\end{table}

